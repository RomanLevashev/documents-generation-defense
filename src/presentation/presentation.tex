\documentclass{beamer}
%Для защит онлайн лучше использовать разрешение 16x9
%\documentclass[aspectratio=169]{beamer}

\input{preamble.tex}

% То, что в квадратных скобках, отображается внизу по центру каждого слайда. 
\title[Документы для защит ВКР]{Генерация документов для защит ВКР в Practice Grading}

% То, что в квадратных скобках, отображается в левом нижнем углу. 
\institute[СПбГУ]{}

% То, что в квадратных скобках, отображается в левом нижнем углу.
\author[Левашев]{Левашев Роман Владимирович, группа 2024.Б.Б44-мм}
 
\begin{document}
{
\setbeamertemplate{footline}{}
% Лого университета или организации, отображается в шапке титульного листа
\begin{frame}
  \includegraphics[width=1.4cm]{pictures/SPbGU_Logo.png}
\vspace{-35pt}
\hspace{-10pt}
\begin{center}
   \begin{tabular}{c}
        \scriptsize{Санкт-Петербургский государственный университет} \\
        \scriptsize{Кафедра системного программирования}
    \end{tabular}
\titlepage
\end{center}

\btVFill

{\scriptsize
  % У научного руководителя должна быть указана научная степень
   \textbf{Научный руководитель:} к. т. н. Ю. B. Литвинов, старший преподаватель кафедры системного программирования \\
  % Консультанта может и не быть. Должна быть указана должность или ученая степень
   \textbf{Консультант:} Мясникова Мария Геннадьевна, АО Нэксайн, инженер-программист
  % Для курсовой не обязателен. Должна быть указана должность или ученая степень
%   \textbf{Рецензент:} 
 }
\begin{center}
  \vspace{5pt}
  \scriptsize{Санкт-Петербург\\
                 2025}
  \end{center}

\end{frame}
}

\begin{frame}[fragile]
  \frametitle{Введение}
  \begin{itemize}
    \item \textbf{PracticeGrading} --- веб-приложение для подготовки заседаний комиссии и проведения защит
    \item Автоматизирует формирование пакета документов для секретаря комиссии
    \item Проблемы исходной версии:
	\begin{itemize}
          \item не формировался полный пакет документов
          \item в модели данных не хватало сведений для генерации части документов
          \item часть шаблонов не соответствовала новому стандарту оформления
	\end{itemize}
  \end{itemize}
\end{frame}
            
\begin{frame}
  \frametitle{Существующие решения, вывод и направление решения}
  \begin{itemize}
    \item Ручное оформление (Word/Excel)
      \begin{itemize}
        \item \textit{Минусы:} копирование данных между файлами, расхождения версий, высокий риск ошибок
      \end{itemize}

    \item Универсальные заполнители шаблонов (Form Pilot и аналоги)
      \begin{itemize}
        \item \textit{Минусы:} нет предметной модели ГЭК $\Rightarrow$ данные часто приходится вводить вручную для каждого документа
      \end{itemize}

    \item Вывод: нужна система, которая автоматически формирует полный набор документов в актуальном стандарте
    \item Направление решения: централизованный ввод данных в PracticeGrading $\rightarrow$ генерация пакета документов,
          меньше рутины и ниже риск ошибок
  \end{itemize}
\end{frame}


% Обязательный слайд: четкая формулировка цели данной работы и постановка задачи
% Описание выносимых на защиту результатов, процесса или особенностей их достижения и т.д.
\begin{frame}
  \frametitle{Постановка задачи}
  \begin{itemize}
  \item Расширить модель данных и реализовать веб-интерфейс для управления членами комиссии
    \item Доработать генерацию документов:
	\begin{itemize}
	  \item Привести генерацию уже формируемых документов к новому стандарту
	  \item Реализовать генерацию недостающих документов
      \end{itemize}

    \item Протестировать изменения:
      \begin{itemize}
        \item e2e-тесты сценариев управления членами комиссии 
        \item проверка корректности генерации документов 
      \end{itemize}
  \end{itemize}
\end{frame}
            
%Идеально, если есть по одному слайду на каждую поставленную задачу            

\begin{frame}
  \frametitle{Расширение модели данных и веб-интерфейс для управления членами комиссии}
  \begin{itemize}
    \item Расширена модель данных для хранения сведений о членах комиссии, необходимых для заполнения документов
    \item Реализован веб-интерфейс для управления членами комиссии:
    \item На backend добавлена поддержка новой функциональности:
      \begin{itemize}
        \item API-эндпоинты для операций управления членами комиссии
        \item доработки работы с БД для хранения и обновления данных
      \end{itemize}
  \end{itemize}
\end{frame}

\begin{frame}{Веб-интерфейс управления членами комиссии}
  \begin{columns}[T,onlytextwidth]
    \begin{column}{0.68\textwidth}
      \centering
      \includegraphics[
        width=\linewidth,
        height=0.78\textheight,
        keepaspectratio
      ]{pictures/commission-list.png}

      \vspace{0.2em}
      \tiny Список, поиск и добавление членов комиссии
    \end{column}

    \begin{column}{0.32\textwidth}
      \centering
      \includegraphics[
        width=\linewidth,
        height=0.78\textheight,
        keepaspectratio
      ]{pictures/commission-edit.png}

      \vspace{0.2em}
      \tiny Форма редактирования сведений о члене комиссии
    \end{column}
  \end{columns}
\end{frame}

\begin{frame}{Генерация документов}
  \begin{columns}[T,onlytextwidth]
    \begin{column}{0.58\textwidth}
      \begin{itemize}
        \item Алгоритм работы:
            \begin{enumerate}
            \item выбирается шаблон документа
            \item плейсхолдеры заполняются данными
            \item формируются таблицы и списки (при необходимости)
            \item сохраняется итоговый файл
            \end{enumerate}
        \item Для работы с шаблонами Microsoft Office используется библиотека NPOI
      \end{itemize}
    \end{column}

    \begin{column}{0.42\textwidth}
      \centering
      \includegraphics[
        width=\linewidth,
        height=0.78\textheight,
        keepaspectratio
      ]{pictures/statement-template.png}

      \vspace{0.2em}
      \tiny Пример шаблона ведомости
    \end{column}
  \end{columns}
\end{frame}

\begin{frame}{Схема генерации документов}
\centering
\includegraphics[width=\linewidth]{pictures/scheme.pdf}
\end{frame}

\begin{frame}
  \frametitle{Тестирование}
  \begin{itemize}
    \item Обновлены существующие автотесты под новую модель данных
    \item Добавлены проверки:
      \begin{itemize}
        \item UI-тесты Playwright: покрыта новая функциональность по добавлению и управлению членами комиссии
        \item backend-тесты API: проверены новые методы, реализованные для поддержки веб-интерфейса
        \item тесты генерации документов: корректность сформированного пакета подтверждается сравнением с эталонными документами
      \end{itemize}
    \item Результат: все обновлённые и добавленные тесты успешно проходят
  \end{itemize}
\end{frame}

\begin{frame}
  \frametitle{Результаты}
  \begin{itemize}
    \item Расширена модель данных: добавлены сведения, необходимые для генерации документов (в т.\,ч. по членам комиссии)
    \item Реализован веб-интерфейс для управления составом комиссии
    \item Генерация документов доработана:
      \begin{itemize}
        \item обновлена генерация ранее формируемых документов под новый стандарт: \emph{ведомость}, \emph{оценочные листы членов ГЭК}
        \item реализована генерация новых документов:
          \begin{itemize}
            \item согласие на обработку персональных данных членов ГЭК
            \item итоговый протокол
            \item протоколы защиты
            \item отчёт председателя
          \end{itemize}
      \end{itemize}
  \end{itemize}
\end{frame}
\end{document}
