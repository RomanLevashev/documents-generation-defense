% Данный файл распространяетсяя под лицензией CC BY 4.0. Текст лицензии размещён на https://creativecommons.org/licenses/by/4.0/
% (c) Кафедра системного программирования, 2025

\documentclass[a4paper]{article}

\usepackage[a4paper, top=8mm, bottom=8mm, left=8mm, right=8mm]{geometry}

\usepackage{polyglossia}
\setdefaultlanguage[babelshorthands=true]{russian}
\setotherlanguage{english}

\usepackage{fontspec}
\setmainfont{FreeSerif}
\newfontfamily{\russianfonttt}[Scale=0.7]{DejaVuSansMono}

\usepackage[tiny, compact]{titlesec}

\usepackage{titling}
\setlength{\droptitle}{-1cm}
\pretitle{\begin{center}\begin{bfseries}\Large}
\posttitle{\par\end{bfseries}\end{center}}
\preauthor{\begin{center}\normalsize}
\postauthor{\par\end{center}\vspace{-1.8cm}}

\usepackage{hyperref}
\usepackage{bookmark}
\usepackage{csquotes}

\title{Генерация документов для защит ВКР в Practice Grading}

\author{Левашев Роман Владимирович}

\date{}

\begin{document}

\maketitle

\begin{flushright}
    Группа: \emph{2024.Б.Б44-мм}

    Кафедра: \emph{Системного Программирования}

    Научный руководитель: \emph{Литвинов Юрий Викторович}

    Консультант: \emph{Мясникова Мария}

    Номер семестра практики: \emph{3}
\end{flushright}

\section{Постановка задачи}

Работа выполнялась в рамках существующего проекта \textbf{PracticeGrading} --- веб-приложения
для автоматизации подготовки заседаний комиссии и проведения защит.
Одной из возможностей приложения является генерация документов, необходимых для проведения заседания,
однако в исходной реализации поддерживалась генерация не всех требуемых документов, а часть шаблонов не соответствовала актуальному стандарту оформления.

Цель работы --- обновить генерацию ранее существующих документов под новый стандарт и реализовать генерацию недостающих документов.
Результат ориентирован на секретаря комиссии и направлен на сокращение рутинных операций,
времени подготовки и риска ошибок при оформлении документации.

\section{Описание предлагаемого решения}

Для генерации части документов в исходной реализации не хватало данных о членах комиссии, поэтому была доработана схема базы данных и реализована функциональность для удобного управления членами комиссии (добавление, поиск, редактирование и удаление), обеспечивающая хранение всей необходимой информации для заполнения шаблонов.

Генерация ранее существующих документов была доработана в соответствии с новым стандартом оформления,
а также реализована генерация недостающих документов, необходимых для проведения заседаний (см. перечень в заключении).

Проект использует технологический стек: backend на \textbf{C\#} (\textbf{ASP.NET Core}) и frontend на \textbf{TypeScript} (\textbf{React}).
Модуль генерации документов реализован на стороне backend и использует библиотеку для работы с форматами Microsoft Office
\textbf{NPOI} для формирования и заполнения файлов по шаблонам.

\textbf{Краткий обзор аналогов.}
Косвенными аналогами являются универсальные заполнители форм (например, Form Pilot): они позволяют заполнять готовые бланки,
но не содержат предметной модели ГЭК и, как правило, требуют ручного внесения данных в каждый документ.
\textbf{PracticeGrading же изначально ориентирован на регламент и формат проведения заседаний в СПбГУ},
поэтому генерация документов логично опирается на встроенную предметную модель и данные, вводимые централизованно через веб-интерфейс. Это снижает долю ручного оформления и уменьшает риск ошибок при подготовке итоговых файлов.

\section{Тестирование}

Для адаптации проекта под новую модель данных были обновлены существующие автоматические тесты.
Также добавлены новые проверки, покрывающие управление членами комиссии и генерацию документов.
\begin{itemize}
    \item \textbf{обновлены} имеющиеся UI-тесты на Playwright и backend-тесты, чтобы они корректно работали с обновлённой моделью данных;
    \item \textbf{добавлены} UI-тесты на Playwright, проверяющие через интерфейс добавление и редактирование членов комиссии;
    \item \textbf{добавлены} backend-тесты API, проверяющие добавление, редактирование и поиск членов комиссии, а также корректность взаимодействия API с БД;
    \item \textbf{добавлены} тесты генерации документов: сформированные файлы сравниваются с эталонными версиями.
\end{itemize}

Корректность доработок подтверждена: \textbf{все обновлённые и добавленные тесты успешно проходят}.

\section{Заключение}

В ходе выполнения работы получены следующие результаты:
\begin{itemize}
    \item добавлена функциональность хранения информации о членах комиссии;
    \item \textbf{переделана} генерация документов: \emph{ведомость}, \emph{оценочные листы членов ГЭК};
    \item \textbf{добавлена} генерация документов: \emph{согласие на обработку персональных данных членов ГЭК}, \emph{итоговый протокол},
          \emph{протоколы защиты}, \emph{отчёт председателя}.
\end{itemize}

Материалы, иллюстрирующие выполнение работы: \url{https://github.com/RomanLevashev/documents-generation-defense/tree/main/demonstration}.

Репозиторий/ссылка на пуллреквест: \url{https://github.com/yurii-litvinov/PracticeGrading/pull/11}.

Техническая документация: \url{https://github.com/RomanLevashev/documents-generation-defense/blob/main/tech_doc.pdf}.

Статус: \emph{Решение прошло процесс ревью}.

\end{document}
